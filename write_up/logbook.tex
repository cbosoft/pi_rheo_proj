\documentclass[a4]{report}
\def\atitle{Logbook}
\def\myname{Christopher Boyle}

% Preamble
\usepackage{graphicx}
\usepackage{geometry}
\geometry{top=2cm, bottom=2cm, left=3cm, right=3cm}

\begin{document}
	\begin{titlepage}
		\makebox[\textwidth][c]{\includegraphics[scale=1]{images/titleheader.png}}
		\centering
		\vskip4cm
		{
			\bfseries\Large
			Department of Chemical \& Process Engineering\\
			\vskip1cm
			MEng in Chemical \& Process Engineering\\
			18530
			\vskip3cm
			\LARGE\atitle
		}
		\vskip3cm
		\begin{flushleft}
			\vskip3cm
			Your name: \myname \hfill Date: \today
			\vskip1cm
			Organisation: University of Strathclyde, Department of Chemical \& Process Engineering\newline% \newline
			In-house Supervisor: Dr. Leo Lue \newline% \newline
			Academic Supervisor:  Dr. Leo Lue
		\end{flushleft}
	\end{titlepage}
	
	\tableofcontents
	
	\chapter{December 2016}
	\section{Week Beginning 5 December 2016 (Week -5)}
	\begin{itemize}
		\item Went to meet with Dr. Lue to discuss the details of the LL3 research project. 
		\item Begun preliminary research on colloidal suspensions, jamming, using the Raspberry Pi, electronic circuits.
	\end{itemize}
	\section{Up to Week Beginning 9 January 2017 (Weeks -5 to 0)}
	\begin{itemize}
		\item Christmas break
		\item Continued background reading and research. 
		\item Begun work on python scripts.
		\item Begun writing the web control interface.
		\item Begun learning how to use \LaTeX
		\item Begun writing basic structure and format of report.
	\end{itemize}
	\chapter{January 2017}
	\section{Week Beginning 9 January (Week 0)}
	\begin{itemize}
		\subsection*{Monday}
		\item Met with Dr. Lue.
		\item Had a tour about the lab, looked at the hardware that would be used.
		\item Discussed what steps should first be taken (focus on how to detect the motor's speed), and the format of contact (regular email contact, meetings as necessary).
		\item Discussed methods of calibrating the motor speed detector (relating speed to voltage, torque to current, etc)
		\item Found a method for speed detection using magnets and a Hall Effect sensor.
		\subsection*{Wednesday}
		\item Dr. Lue postulated another method for detecting the speed using a light sensor, some reflective material mounted on the rotating cylinder and a laser.
		\subsection*{Friday}
		\item Discovered a simple electronic circuit using a potentiometer/voltage-divider and a transistor to control the voltage being supplied to the motor (to control the speed)
		\item Implemented a scaled-down prototype version of the speed control circuit, it seemed to work well.
		\subsection*{Saturday}
		\item Attempted to implement a full scale version of the speed control circuit, but was unable to accurately solder the digital potentiometer due to its size (around 3mm x 1mm). Alternative component sought.
	\end{itemize}
	\newpage
	\section{Week beginning 16 January 2016 (Week 1)}
	\begin{itemize}
		\subsection*{Monday}
		\item Met with Dr. Lue and Dr. Haw, RE motor speed control.
		\item Discussed different methods of calibrating the motor, decided to use a reference fluid to test how well the RPi system works vs a commercial rheometer.
		\item  To do this week: fill out risk assessment, find out more about the Hall Effect sensor as a means to detect speed, set up speed control and speed detection circuit, and get it working with the pi.
		\subsection*{Tuesday}
		\item Found problem with the hall effect sensor: magnetic field in single small magnet not enough to trigger it. Two magnets required, possible creating problems when fitting later on?
		\item Hall effect sensor not working properly with the raspberry pi - it is pulling down the pin voltage too far: the RPi cannot sense the signal.
		\item Dr. Lue gave me some smaller magnets, will test these with the hall effect sensor.
		\item Drew up draft risk assessment for working in the lab.
		\item Fixed problem with hall effect sensor circuit (by adding a transistor)
		\item Hall sensor is triggered when a magnet's south pole comes within 10mm of the sensor.
		\item To do tomorrow: go into lab and test speed sensor with the actual motor, check motor class is working properly
	\end{itemize}
	
\end{document}