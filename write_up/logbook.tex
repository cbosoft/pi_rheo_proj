\documentclass[a4]{report}
\def\atitle{Logbook}
\def\myname{Christopher Boyle}

% Preamble
\usepackage{graphicx}
\usepackage{geometry}
\geometry{top=2cm, bottom=2cm, left=3cm, right=3cm}

\begin{document}
	\begin{titlepage}
		\makebox[\textwidth][c]{\includegraphics[scale=1]{images/titleheader.png}}
		\centering
		\vskip4cm
		{
			\bfseries\Large
			Department of Chemical \& Process Engineering\\
			\vskip1cm
			MEng in Chemical \& Process Engineering\\
			18530
			\vskip3cm
			\LARGE\atitle
		}
		\vskip3cm
		\begin{flushleft}
			This project is submitted in partial fulfillment of the regulations governing the award of \\
			Degree of MEng in Chemical Engineering at the University of Strathclyde
			\vskip2cm
			Your name: \myname \hfill Date: \today
			\vskip1cm
			Organisation: University of Strathclyde, Department of Chemical \& Process Engineering\newline% \newline
			In-house Supervisor: Dr. Leo Lue \newline% \newline
			Academic Supervisor: Dr. Leo Lue
		\end{flushleft}
	\end{titlepage}
	
	\tableofcontents
	
	\chapter{December 2016}
	\section{Week Beginning 5 December 2016 (Week -5)}
	\begin{itemize}
		\item Went to meet with Dr. Lue to discuss the details of the LL3 research project. 
		\item Begun preliminary research on colloidal suspensions, jamming, using the Raspberry Pi, electronic circuits.
	\end{itemize}
	\section{Up to Week Beginning 9 January 2017 (Weeks -5 to 0)}
	\begin{itemize}
		\item Christmas break
		\item Continued background reading and research. 
		\item Begun work on python scripts.
		\item Begun writing the web control interface.
		\item Begun learning how to use \LaTeX
		\item Begun writing basic structure and format of report.
	\end{itemize}
	\chapter{January 2017}
	\section{Week Beginning 9 January (Week 0)}
	\begin{itemize}
		\subsection*{Monday}
		\item Met with Dr. Lue.
		\item Had a tour about the lab, looked at the hardware that would be used.
		\item Discussed what steps should first be taken (focus on how to detect the motor's speed), and the format of contact (regular email contact, meetings as necessary).
		\item Discussed methods of calibrating the motor speed detector (relating speed to voltage, torque to current, etc)
		\item Found a method for speed detection using magnets and a Hall Effect sensor.
		\subsection*{Wednesday}
		\item Dr. Lue postulated another method for detecting the speed using a light sensor, some reflective material mounted on the rotating cylinder and a laser.
		\subsection*{Friday}
		\item Discovered a simple electronic circuit using a potentiometer/voltage-divider and a transistor to control the voltage being supplied to the motor (to control the speed)
		\item Implemented a scaled-down prototype version of the speed control circuit, it seemed to work well.
		\subsection*{Saturday}
		\item Attempted to implement a full scale version of the speed control circuit, but was unable to accurately solder the components required. Different components were sought.
	\end{itemize}
	\newpage
	\section{Week beginning 16 January 2016 (Week 1)}
	\begin{itemize}
		\subsection*{Monday}
		\item Met with Dr. Lue and Dr. Haw, RE motor speed control.
		\item discussed different methods of calibrating the motor. Decided to use a reference fluid to test how well the RPi system works vs a commercial rheometer.
		\item  To do this week: find out more about the Hall Effect sensor as a means to detect speed, set up speed control and speed detection circuit and get it working with the pi.
	\end{itemize}
	
\end{document}