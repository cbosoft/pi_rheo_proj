\documentclass{report}

\usepackage{geometry}

\geometry{left=25mm, right=25mm, top=30mm, bottom=30mm}

\begin{document}
	
	{
		\pagenumbering{gobble}
		\centering
		\Huge \noindent
		\texttt{RPi-R}
		\vskip2cm
		\normalsize Using the Taylor-Couette Cell Rotational Rheometer with the Raspberry Pi
		\vfill
		\small Christopher Boyle \hfill \today
		\newpage
	}
	
	\tableofcontents
	
	\chapter{Overview}
	
		\section{Hardware}
			This section describes the main hardware that comprises the rheometer; the Taylor-couette cell, the bench, the scissor lift, etc. Most (i.e. all) of this was not of my design or choosing, it was set up by the department for a previous set of students, therefore I don't know the exact reasoning behind the choices, but I'll give it a go.
			
			The Taylor-Couette cell is composed of a hollow glass cylinder and a perspex inner cylinder. The inner cylinder is attached to the rotor of a DC motor. The outer cylinder is attached to a scissor lift. Fluid can be placed into the gap between the cylinders and sheared - this is roughly the whole rheometer. In addition, there are sensors and other electronics which allow the hardware to be controlled by a Raspberry Pi computer, discussed in the next section.
			
			The cell is based upon a heavy metal base, of unsure exact function. It is a nice way to organise the rheometer, it provides a stable base, and it protects the worktop, but a bare worktop would have been fine too. On this bench sits the scissor lift holding the fixed outer cylinder, as well as a clamp stand which holds the motor (and the inner cylinder).
			
			The motor is a 15V DC electric motor with a 1/16th reduction gear system attached. This motor can operate nominally with a voltage supply between 4.5V and 15V.
			
			There are some important geometries used in the rheometry calculation; the diameters of the inner and outer cylinders. These are summarised in Table \ref{tablegeoms}.
			
			\begin{table}
				\centering
				\caption{Summary of Hardware Geometries}
				\label{tablegeoms}
				\begin{tabular}{| c | c | c |}
					\hline
					\textbf{Dimension} & \textbf{Symbol} & \textbf{Value} \\ \hline
					Inner cylinder diameter & $\rm D_{IC}$ & m \\ \hline
					Outer cylinder diameter (outer) & $\rm D_{OCo}$ & m \\ \hline
					Outer cylinder diameter (inner) & $\rm D_{OCi}$ & m \\ \hline
				\end{tabular}
			\end{table}
			
			\subsection*{Known Issues}
				\textbf{Assumption:} Couette flow in between the cylinders. \textbf{Strengths:} the cylinders have a small gap, and can be easily adjusted to minimise errors in centring. \textbf{Weaknesses:} both the inner and outer cylinders have small imperfections which could affect the flow in the cell. The glass outer cylinder is non uniform --- to be expected with glass objects. Also, the inner cylinder is slightly off centre --- slightly.
				
				\textbf{Assumption:} motor gear system will not greatly affect the system, or it will but in a way that can be accounted for in calibrations. \textbf{Strengths:} this should be true for any well maintained gear system. Gears only transfer movement and should not present any resistance that is not easily accounted for. \textbf{Weaknesses:} the gear system/motor sometimes acts oddle --- making strange sounds and appearing to slow despite no changes to load or to power supply. This was remedied (hopefully) by applying WD40 to the gear system.
			
		\section{Electronics}
		
		\section{Software}
	
	\chapter{Usage}
	
		\section{Using the rheometer script}
		
		\section{Custom usage scripts}
	
	\chapter{}
	
	\chapter{Why?}

	

\end{document}